% Created 2021-03-24 Wed 12:55
% Intended LaTeX compiler: pdflatex
\documentclass[bigger]{beamer}
\usepackage[utf8]{inputenc}
\usepackage[T1]{fontenc}
\usepackage{graphicx}
\usepackage{grffile}
\usepackage{longtable}
\usepackage{wrapfig}
\usepackage{rotating}
\usepackage[normalem]{ulem}
\usepackage{amsmath}
\usepackage{textcomp}
\usepackage{amssymb}
\usepackage{capt-of}
\usepackage{hyperref}
\usetheme[progressbar=foot]{metropolis}
\author{Edmund Miller}
\date{2021-03-09 Wed}
\title{Genome-wide maps of chromatin state in pluripotent and lineage-committed cells}
\hypersetup{
 pdfauthor={Edmund Miller},
 pdftitle={Genome-wide maps of chromatin state in pluripotent and lineage-committed cells},
 pdfkeywords={\href{../../concepts/biology/sequencing/chip\_seq.org}{ChIP-seq}},
 pdfsubject={a framework for deterministic machine learning},
 pdfcreator={Emacs 28.0.50 (Org mode 9.5)}, 
 pdflang={English}}
\begin{document}

\maketitle

\section{Background}
\label{sec:org49fde6a}

\begin{frame}[label={sec:org34e898f}]{Histones}
\begin{center}
\includegraphics[height=0.7\linewidth]{/home/emiller/sync/org/roam/data/7b/330aee-4d85-4681-a809-025f5db75e13/_20210324_114732screenshot.png}
\end{center}
\end{frame}

\begin{frame}[label={sec:orgffa065f}]{Histone modifications}
\begin{center}
\includegraphics[height=0.7\linewidth]{/home/emiller/sync/org/roam/data/86/f221d7-720c-47ff-9df2-4ef659272ff0/_20210324_115719screenshot.png}
\end{center}
\end{frame}


\begin{frame}[label={sec:org70117aa}]{Histone modifications}
\begin{center}
\includegraphics[width=.9\linewidth]{/home/emiller/sync/org/roam/data/b0/9008f9-aa29-47b9-ae53-b747ff67ba17/_20210324_115559screenshot.png}
\end{center}

\begin{itemize}
\item Bivalent Chromatin
\end{itemize}
\end{frame}

\begin{frame}[label={sec:org9917a63}]{Chromatin immunoprecipitation (ChIP)}
\begin{enumerate}
\item DNA and proteins are crosslinked
\item The DNA-protein complexes are sheared (sonication or nuclease digestion)
\item Proteins are pulled out using antibodies, along with the cross-linked DNA fragments
\item DNA fragments are purified
\end{enumerate}
\end{frame}

\begin{frame}[label={sec:org7d9fcf5}]{Sequencing technology}
\begin{center}
\includegraphics[width=.9\linewidth]{/home/emiller/sync/org/roam/data/04/540cc4-c1f8-4b34-b004-6f92fd9c71ad/_20210324_120958screenshot.png}
\end{center}

(Hass and Zody, Advancing RNA-Seq analysis, Nature Biotechnology 28:421-423)
\end{frame}
\begin{frame}[label={sec:org0fd816b}]{ChIP-seq}
\begin{center}
\includegraphics[width=.9\linewidth]{/home/emiller/sync/org/roam/data/e6/d26fa2-787e-4872-abf8-f9f649fe40ad/_20210324_121550screenshot.png}
\end{center}
\end{frame}

\begin{frame}[label={sec:org43e8a90}]{CpG Sites}
\begin{center}
\includegraphics[width=.9\linewidth]{/home/emiller/sync/org/roam/data/62/3ea5dc-1211-4695-96d9-362097364d9e/_20210324_091853screenshot.png}
\end{center}
\end{frame}



\section{Introduction}
\label{sec:org2c9dc3f}
\begin{frame}[label={sec:orge8fb555}]{History}
\begin{itemize}
\item Mikkelson et al. from the Broad submitted to Nature and was published in August 2007.
\item Johnson et al. from Stanford submitted to Science and was published in June 2007.
\item Barski et al. from NHLBI, NIH submitted to Cell and was published in May 2007.

\item Barski was the third most cited biology paper in 2009

\item All of these were after Frank Pugh’s work on H2A.Z in yeast
\end{itemize}

\href{http://www.snarkyscientist.com/2013/06/19/the-history-of-the-biggest-technique-of-2009-who-invented-chip-seq/}{Snarky Scientist}
\end{frame}

\begin{frame}[label={sec:org8f4a699}]{Goals}
\begin{itemize}
\item Report the development of a method for mapping ChIP enrichment by sequencing
(ChIP-Seq)
\item Define Broad categories of promoters based on their chromatin state and create maps
\item Demonstrate using ChIP for genome-wide annotation of novel promoters
\end{itemize}
\end{frame}

\begin{frame}[label={sec:org22d436b}]{Histone Marks}
\begin{block}{Actively transcribed genes}
\begin{itemize}
\item H3K4me3 - Promoters
\item H3K36me3 - Body of active genes
\end{itemize}
\end{block}

\begin{block}{Repressed Genes}
\begin{itemize}
\item H3K27me3 - Gene repression
\item H4K20me3 - Tightly associated with heterochromatin
\item H3K9me2/3 - Heterochromatin and Gene repression
\end{itemize}
\end{block}

\begin{block}{Bivalent Promoters}
\begin{itemize}
\item Both H3K4Me3 and H3K27Me3
\end{itemize}
\end{block}
\end{frame}

\section{Genome-wide chromatin-state maps}
\label{sec:orgff98a94}
\begin{frame}[label={sec:org58b31a9},shrink=10]{Supplementary Table 1}
\begin{center}
\begin{tabular}{llrl}
Cell type1 & Epitope & Replicates & Uniquely aligned reads\\
\hline
ES cells & pan-H3 & 1 & 4,490,474\\
 & H3K4me3 & 2 & 8,398,790\\
 & H3K9me3 & 2 & 4,411,447\\
 & H3K27me3 & 2 & 7,211,279\\
 & H3K36me3 & 2 & 7,217,118\\
 & H4K20me3 & 2 & 5,139,339\\
 & RNAP II & 1 & 2,736,500\\
\hline
ES cells- hybrid & H3K4me3 & 2 & 11,471,501\\
 & H3K9me3 & 1 & 3,741,367\\
 & H3K36me3 & 2 & 7,826,112\\
\hline
Neural progenitor cells (NPCs) & H3K4me3 & 2 & 6,995,068\\
 & H3K9me3 & 2 & 4,614,191\\
 & H3K27me3 & 2 & 8,166,774\\
 & H3K36me3 & 2 & 7,899,115\\
\hline
Embryonics fibroblasts (MEFs) & H3K4me3 & 2 & 11,371,374\\
Lineage-commited & H3K9me3 & 2 & 4,468,908\\
 & H3K27me3 & 2 & 12,208,145\\
 & H3K36me3 & 2 & 10,315,848\\
\end{tabular}
\end{center}
\end{frame}

\begin{frame}[label={sec:orge45a040}]{Figure 1 | Comparison of ChIP-Seq and ChIP-chip data}
\begin{center}
\includegraphics[width=1.0\linewidth]{/home/emiller/sync/org/roam/data/4f/4a18ac-12d3-40ce-a681-6184cbf3faff/_20210323_210938screenshot.png}
\end{center}
\end{frame}

\begin{frame}[label={sec:orgc520755}]{Supplementary Figure 2}
\begin{center}
\includegraphics[width=1.0\linewidth]{/home/emiller/sync/org/roam/data/f5/d9faad-eda5-4ff5-9bab-ed6df6ac3d8f/_20210323_213424screenshot.png}
\end{center}
\end{frame}

\section{Promoter state in ES and lineage-committed cells}
\label{sec:orgea3e35f}
\begin{frame}[label={sec:orgaadc469}]{High CpG promoters in ES cells}
\begin{itemize}
\item Divided promoters into 'high' CpG \href{../../concepts/biology/promoter.org}{promoters}(HCP) and 'low' groups, and an
intermediate. 11K high, 3K low and 3.3K intermediate
\end{itemize}
\begin{itemize}
\item HCP Associated with significant \href{../../concepts/biology/h3k4me3.org}{H3K4me3} within a 1-2kb window
\item There is a strong correlation between \href{../../concepts/biology/h3k4me3.org}{H3K4me3} intensity and the expression level of associates genes
\item Not all \href{../../concepts/biology/promoter.org}{promoters} associated with \href{../../concepts/biology/h3k4me3.org}{H3K4me3} are active.
\end{itemize}
\end{frame}

\begin{frame}[label={sec:org06f563f}]{High CpG promoters in ES cells}
\begin{itemize}
\item \textasciitilde{}22\% of HCPs are bivalent and show both \href{../../concepts/biology/h3k4me3.org}{H3K4me3} and H3K27me3 (Fig 2b)
\begin{itemize}
\item The majority are 'narrow' with more punctate H3K27me3
\end{itemize}

\item Bivalent Promoters show low activity despite \href{../../concepts/biology/h3k4me3.org}{H3K4me3}
\begin{itemize}
\item This suggests that the represseive PcG activity dominates the ubiquitous \href{../../concepts/biology/trxg.org}{trxG} activity
\end{itemize}

\item Monovalent promoters \href{../../concepts/biology/h3k4me3.org}{(H3K4me3}) regulate housekeeping genes, they're just always on
\item \href{../../concepts/biology/bivalent\_chromatin.org}{Bivalent} \href{../../concepts/biology/promoter.org}{promoters} are associated with genes with more complex expression
patterns
\begin{itemize}
\item Developmental \href{../../concepts/biology/transcription\_factor\_tf.org}{transcription factors}
\item Morphogens
\item Cell surface molecules
\item Lineage-specific microRNAs
\end{itemize}
\end{itemize}
\end{frame}

\begin{frame}[label={sec:org50e451d}]{Figure 2 | Histone trimethylation state predicts expression of HCPs andLCPs}
\begin{center}
\includegraphics[height=0.7\linewidth]{/home/emiller/sync/org/roam/data/c0/805681-907a-498f-b20f-a4a26703b9ac/_20210324_110804screenshot.png}
\end{center}
\end{frame}


\begin{frame}[label={sec:org342f446}]{High CpG promoters in NPCs and MEFs}
\begin{itemize}
\item Most HCPs marked with H3K4me3 alone in ES cells retain this mark both in NPCs
and MEFs
\begin{itemize}
\item This sub-class of promoters regulates ubiquitous house-keeping genes
\end{itemize}
\item Small proportion (\textasciitilde{}4\%) of these promoters have H3K27me3 in MEFs
\begin{itemize}
\item Correlates with lower expression levels
\end{itemize}
\end{itemize}
\end{frame}

\begin{frame}[label={sec:org32f0b8b}]{Figure 3 | Cell-type-specific chromatin marks at promoter}
\begin{center}
\includegraphics[height=0.325\linewidth]{/home/emiller/sync/org/roam/data/cd/f6fab7-f85d-4011-8854-1f9fcafb3ef2/_20210324_110905screenshot.png}
\end{center}
\end{frame}


\begin{frame}[label={sec:org71a6e95}]{Figure 3 | Cell-type-specific chromatin marks at promoter}
\begin{center}
\includegraphics[width=1.0\linewidth]{/home/emiller/sync/org/roam/data/3f/f8e55b-d0ad-4fe1-bb72-a0309652830b/_20210324_110925screenshot.png}
\end{center}
\end{frame}


\section{Promoter state reflects lineage commitment and potential}
\label{sec:org535562a}

\begin{frame}[label={sec:org1fbbdd2}]{Figure 4 | Correlation between chromatin-state changes and lineage expression}
\begin{center}
\includegraphics[height=0.7\linewidth]{/home/emiller/sync/org/roam/data/df/3132f4-3345-41a6-b6ec-19a5759ba2c1/_20210324_111024screenshot.png}
\end{center}
\end{frame}


\section{Genome-wide annotation of promoters and primary transcripts}
\label{sec:org4634aa8}

\begin{frame}[label={sec:orga9d165b}]{Figure 5 | H3K4me3 and H3K36me3 annotate genes and non-coding RNAtranscripts.}
\begin{center}
\includegraphics[height=0.7\linewidth]{/home/emiller/sync/org/roam/data/df/714a4b-0442-486d-b12c-513550ed5b35/_20210324_111735screenshot.png}
\end{center}
\end{frame}


\begin{frame}[label={sec:org44a1ca7}]{Figure 6 | Allele-specific histone methylation and genic H3K9me3/H4K20me3}
\begin{center}
\includegraphics[width=.9\linewidth]{/home/emiller/sync/org/roam/data/67/ed5a84-6a68-448c-818a-63ad21906c9f/_20210324_111942screenshot.png}
\end{center}
\end{frame}


\section{H3K9 and H4K20 trimethylation mark specific repetitive elements}
\label{sec:org732be3e}

\begin{frame}[label={sec:org47ded02}]{Figure 6 | Allele-specific histone methylation and genic H3K9me3/H4K20me3}
\begin{center}
\includegraphics[width=1.0\linewidth]{/home/emiller/sync/org/roam/data/79/2b8a6e-36cb-471e-b382-6d0913aba490/_20210324_111952screenshot.png}
\end{center}
\end{frame}

\section{Discussion}
\label{sec:org8bd7349}

\begin{frame}[label={sec:orgddf2db2}]{Discussion}
\begin{itemize}
\item H3K4me3 and H3K36me3 allows recognition of promoters together with their
complete transcription units
\begin{itemize}
\item Should help to define alternative promoters and their usage in specific cell
types
\item Identify the primary structure of genes encoding non-coding RNAs
\item Detect gene expression
\item Detect allele-specific transcription
\end{itemize}

\item H3K9me3 and H4K20me3 should facilitate the study of heterochromatin, spreading
and imprinting mechanisms.
\end{itemize}
\end{frame}

\begin{frame}[label={sec:org945ee15}]{Discussion}
\begin{itemize}
\item H3K4me3 and H3K27me3 provides a rich description of cellular state.
\begin{itemize}
\item \href{../../concepts/biology/promoter.org}{Promoters} may be classified as active, repressed or poised for alternative
developmental fates
\end{itemize}

\item Given the technical features of \href{../../concepts/biology/sequencing/chip\_seq.org}{ChIP-Seq} (high throughput, low cost and input
requirement), it is now appropriate to contemplate projects to generate
catalogues of chromatin-state maps representing a wide range of human and
mouse cell types.
\end{itemize}
\end{frame}
\end{document}